\documentclass{report}
\usepackage{amsmath,tikz,amssymb,stmaryrd,amsfonts}
\usepackage{kotex}
\usepackage{MnSymbol}

\begin{document}
\title{Analysis Specification of YAtern}
\author{Se-Won Kim}
\maketitle

The analysis is basically a pointer analysis.

\chapter{Abstraction of Values}
\section{Abstract Values}
An abstract value for an expression is defined as follows.
\[
A = N \times S \times B \times \powerset(O).
\]
The value of an expression at a specific moment could be a (primitive) number, 
a (primitive) string, a (primitive) boolean or an object.
We collect possible values of an expression according to some criteria 
and abstract into an element of $A$.
The most basic criterion is type:
Therefore, we use 4-ary tuples:
$N$ is the abstract domain for number values, 
$S$ for string values, $B$ for boolean values and $O$ for object values. 

Note that we do not have a domain for the primitive 
value \texttt{undefined}. 
In some sense, any element in $A$ implicitly includes 
the concrete value \texttt{undefined}.
This decision is natural because of the following two reasons.
\begin{itemize}
\item Since our (primary) analysis is flow-insensitive,
  any variable could be \texttt{undefined} before its flow-insensitive use.
\item It is a burden to explicitly represent them in the abstract values.
\end{itemize}
Since \texttt{null} is often interchangable 
with \texttt{undefined} in practice,
we ignore \texttt{null} value, which is of object type,
as we ignore \texttt{undefined} 

\section{Abstraction of Primitive Types}
We can parameterize the abstract domain $A$ by choosing 
different $N$, $S$, $B$ and $O$. 
For our analysis, we use very simple abstract domains for primitive data: 
we abstract the primitive values into their types.
\[
N = \{ \bot_N, \texttt{PrimNumber} \}
\]
where $\bot_N$ represents empty set of numbers and 
\texttt{PrimNumber} represents the set of all numbers. 
We define $S$ and $B$ similarly.
\[
S = \{ \bot_S, \texttt{PrimString} \}
\qquad
B = \{ \bot_B, \texttt{PrimBoolean} \}
\]
$\bot_S$ and $\bot_B$ represent empty sets, and \texttt{PrimString} and 
\texttt{PrimBoolean} represent the sets of all string and booleans, 
accordingly.

\section{Object Types}
ECMAScript 5 does not have a class system for explicit nominal object types. 
Therefore, our analysis infers the structural object types 
from usages of objects.
Since an expression can have values of multiple object types unlike Java,
we use the powerset of $O$ in the definition of $A$.
We define object types, which are structural, as follows.
\[
O = String \rightharpoonup A
\]
where $String$ represents the set of all string values 
and `$\rightharpoonup$' denotes finite partial maps. 

\paragraph{Implementation}
Since structural information could be verbose, 
we will occasionally remember the associated variable or field names 
if possible.
% Then, we can show the object types using the recorded names.
Note that object types are still structural but we can give programmers 
more concise, pseudo nominal information using the recorded names.

\chapter{Abstract Scope}
Functions are a special kind of objects, 
which are first-class values in JavaScript. 
To analyze the functions correctly even when the function's are used in the 
\[
Scope = 
\]
함수 객체의 경우 abstract scope chain을 가지고 다니며 호출된 위치에서 분석해야 한다.
만약 모든 함수를 context insensitive하게 분석한다면 
각 함수마다 하나의 코앞 scope를 가지게 되므로 
scope chain의 구조는 문법적인 nesting 구조와 같다. 
각 함수마다 요약 scope chain의 수는 1개이다.

nested function이 없는 프로그램을 context sensitive하게 분석하면, 
각 함수마다 가능한 요약 scope chain의 갯수는 그 함수의 호출 갯수와 동일하다. 

nested 함수가 있을 때, context sensitive 분석을 하면 
하나의 함수에 대해 좀 더 많은 요약 scope chain이 생길 수 있다.
다음 프로그램을 \texttt{f}는 context sensitive, \texttt{g}는 context insensitive하게
분석한다고 해보자.
\begin{verbatim}
function f(x) {
  function g(y) {
    return x + y;
  }
  return g;
}
var g1 = f(0);
var g2 = f('');
var r1 = g1(0);
var r2 = g2('');
\end{verbatim}
분석 결과는 \texttt{r1}은 \texttt{PrimNumber}, \texttt{r2}는 \texttt{PrimString}으로
나와야 한다. 이 것은 \texttt{g1}, \texttt{g2}가 첫번째 요약 scope는 공유하고,
두번째 요약 scope는 따로 가져야하기 때문이다.

요약 scope가 parent을 가지는 모양이면 위와 같이 sensitivity를 조절할 수 없다. 
대신 요약 함수 값 사이의 chain을 만들어야 할 것이다. 그리고 요약 scope객체는 heap과 유사한
공간에서 다른 객체를 pointing할 수 있게 만든다. 
\begin{itemize}
\item 요약 scope 객체는 parent 대신 outer abstract function object를 가리킨다.
\item abstract function object는 자신의 scope
\end{itemize}


tern은 현재 이런 경우가 생기지 않게 하는 듯하다.
우리도 그다지 필요하지 않을 것 같기는 한데


\section{Function Types}
Functions are special objects in JavaScript. 
We use $F \subseteq O$ to denote the function object types.
They have a few additional abstract values for interprocedural analysis.
For $f \in F$,
\begin{enumerate}
\item $this(f) \in A$ : for values bound to \texttt{this} variable 
  in the function
\item $ret(f) \in A$ : for return value of the function
\item $throw(f) \in A$ : for throwable value of the function
\item $args(f) \in A^*$ : for argments given to the function
\item $scope(f) \in O^*$ : for scope chain of the function
\end{enumerate}


\chapter{Constraint Generation}

\chapter{Basic Constraint Solver}

In this chapter, we show the overall design of the constraint solver. 

One of the most important principles of the solver is that 
we ignore the possibility of \texttt{null} and \texttt{undefined} 
during constraint solving.
Usually using those void values is not the intension of the programmer.
If we take those values into account 
we have much more spurious abstract values.
For example, consider \texttt{c = a + b} where \texttt{a} and \texttt{b} 
are always string values. 
However, the abstractions of the variables contain \texttt{undefined} and
\texttt{undefined + undefined === NaN}, which is of number type. 


\chapter{Context Dependent Analysis}
\begin{enumerate}
\item extend-like functions
\item special case of for-in loops
\end{enumerate}

\chapter{Optimizing the Solver: Forwarding Types}

\end{document}